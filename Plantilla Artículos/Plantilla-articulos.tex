% aps = American Physical Society
% prb = Physical Review B (estilo sólido)
% twocolumn = Doble columna
% superscriptaddress = Para que la uni salga bonita bajo el nombre
\documentclass[aps,prb,twocolumn,superscriptaddress,spanish,10pt]{revtex4-2}

% Cargamos tu preámbulo específico para papers
% --- PAQUETES DE IDIOMA Y CODIFICACIÓN ---
\usepackage[utf8]{inputenc}
\usepackage[T1]{fontenc}
\usepackage[spanish]{babel} % Para que ponga "Cuadro" y "Figura"

% --- MATEMÁTICAS Y FÍSICA (Tus herramientas) ---
\usepackage{amsmath, amssymb, amsthm, mathtools}
\usepackage{physics}  % Imprescindible para tus derivadas \pdv
\usepackage{siunitx}  % Para unidades \SI{3e8}{\meter\per\second}
\usepackage{bm}       % Negritas matemáticas potentes
\usepackage{cancel}

% --- GRÁFICOS Y FIGURAS ---
\usepackage{graphicx}
% \usepackage{float}
\usepackage{booktabs} % Tablas profesionales (sin líneas verticales)

% --- HIPERVÍNCULOS ---
\usepackage{hyperref}
\hypersetup{
    colorlinks=true,
    linkcolor=blue,   % Azul estilo APS
    citecolor=blue,
    urlcolor=blue
}

% --- AJUSTES DE ESPAÑOL EN REVTEX ---
% Revtex a veces se pone tonto con los nombres en inglés
\addto\captionsspanish{
  \renewcommand{\figurename}{FIGURA}
  \renewcommand{\tablename}{TABLA}
  \renewcommand{\refname}{REFERENCIAS}
}

\AtBeginDocument{\RenewCommandCopy\qty\SI}

% Archivo de bibliografía (asegúrate de tener ref.bib)
% Nota: Revtex prefiere BibTeX estándar, no biblatex.
\bibliographystyle{apsrev4-2} 

\begin{document} 

% ==========================================
% BLOQUE DE TÍTULO Y AUTOR (El encabezado Pro)
% ==========================================

\title{Título del Experimento o Investigación: \\ Subtítulo Descriptivo y Técnico}

\author{Marcos Bermúdez Cobarro}
\email{marcos.bermudez@um.es} % Opcional: tu email profesional
\affiliation{Grado en Física, Universidad de Murcia, 30100 Murcia, España.}

\date{\today}

% ==========================================
% RESUMEN (ABSTRACT)
% ==========================================
% En los papers, el resumen va ANTES de imprimir el título
\begin{abstract}
\textbf{Resumen:} En este experimento se ha analizado el comportamiento de... (Aquí describes en 4-5 líneas el objetivo, el método y la conclusión principal). Se ha obtenido un valor experimental para la constante de Planck de $h = \SI{6.62e-34}{\joule\second}$, con un error relativo del \SI{0.5}{\percent} respecto al valor teórico. Los resultados confirman la hipótesis de cuantización.
\end{abstract}

% Imprime todo lo anterior (Título, Autor, Abstract)
\maketitle

% ==========================================
% CONTENIDO DEL ARTÍCULO
% ==========================================

\section{Introducción}
El efecto fotoeléctrico, explicado por Einstein en 1905 \cite{einstein1905}, demuestra la naturaleza corpuscular de la luz. La ecuación fundamental es:
\begin{equation}
    K_{\text{max}} = h\nu - W
\end{equation}
Donde $W$ es la función de trabajo del metal.

\section{Método Experimental}
Se utilizó una lámpara de mercurio y una red de difracción... Como vemos en la Ecuación (1), la relación es lineal.

% EJEMPLO DE FIGURA A UNA COLUMNA
\begin{figure}[t] % [t] = top (arriba). NUNCA uses [h] en doble columna.
    \centering
    \includegraphics[width=0.4\columnwidth]{Img/umulogo.jpg} 
    \caption{Esquema del montaje experimental.}
    \label{fig:montaje}
\end{figure}

\section{Resultados y Discusión}
Los datos obtenidos se muestran en la Tabla \ref{tab:datos}. Nótese el uso de \texttt{booktabs} para tablas elegantes.

\begin{table}[b] % [b] = bottom (abajo)
    \centering
    \caption{Voltaje de frenado vs Frecuencia.}
    \begin{tabular}{cc}
        \toprule
        $\nu$ (\SI{e14}{\hertz}) & $V_0$ (\SI{}{\volt}) \\
        \midrule
        5.49 & 0.65 \\
        6.88 & 1.22 \\
        7.41 & 1.45 \\
        \bottomrule
    \end{tabular}
    \label{tab:datos}
\end{table}

% EJEMPLO DE ECUACIÓN LARGA (MULTLINE)
Si tienes una ecuación enorme que no cabe en una columna, usa \texttt{multline}:
\begin{multline}
    V(r) = -\frac{G M m}{r} + \frac{L^2}{2\mu r^2} \\
    - \frac{G M}{c^2 r^3} (L^2 + \dots)
\end{multline}

\section{Conclusiones}
El experimento ha sido un éxito rotundo. La física funciona.

% ==========================================
% BIBLIOGRAFÍA
% ==========================================
% Aquí le decimos qué archivo .bib usar
\bibliography{ref.bib}

\end{document}