% --- PAQUETES DE IDIOMA Y CODIFICACIÓN ---
\usepackage[utf8]{inputenc}
\usepackage[T1]{fontenc}
\usepackage[spanish]{babel} % Para que ponga "Cuadro" y "Figura"

% --- MATEMÁTICAS Y FÍSICA (Tus herramientas) ---
\usepackage{amsmath, amssymb, amsthm, mathtools}
\usepackage{physics}  % Imprescindible para tus derivadas \pdv
\usepackage{siunitx}  % Para unidades \SI{3e8}{\meter\per\second}
\usepackage{bm}       % Negritas matemáticas potentes
\usepackage{cancel}

% --- GRÁFICOS Y FIGURAS ---
\usepackage{graphicx}
% \usepackage{float}
\usepackage{booktabs} % Tablas profesionales (sin líneas verticales)

% --- HIPERVÍNCULOS ---
\usepackage{hyperref}
\hypersetup{
    colorlinks=true,
    linkcolor=blue,   % Azul estilo APS
    citecolor=blue,
    urlcolor=blue
}

% --- AJUSTES DE ESPAÑOL EN REVTEX ---
% Revtex a veces se pone tonto con los nombres en inglés
\addto\captionsspanish{
  \renewcommand{\figurename}{FIGURA}
  \renewcommand{\tablename}{TABLA}
  \renewcommand{\refname}{REFERENCIAS}
}

\AtBeginDocument{\RenewCommandCopy\qty\SI}