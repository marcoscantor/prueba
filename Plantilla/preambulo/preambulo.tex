% --- 1. CODIFICACIÓN E IDIOMA ---
\usepackage[utf8]{inputenc}
\usepackage[T1]{fontenc}
\usepackage[spanish, es-tabla]{babel} 
\usepackage{csquotes} 

% --- 2. DISEÑO DE PÁGINA ---
\usepackage[a4paper, margin=2.5cm]{geometry}
\usepackage{setspace} 
\usepackage{fancyhdr} 
\usepackage{float}    
\usepackage{enumitem} 

% --- 3. MATEMÁTICAS Y FÍSICA ---
\usepackage{amsmath, amssymb, amsthm, mathtools}
\usepackage{tensor}   
\usepackage{cancel}   
\usepackage{siunitx}
\sisetup{per-mode=symbol} 

% Paquete Physics (Siempre después de amsmath)
\usepackage[italicdiff]{physics} 
\AtBeginDocument{\RenewCommandCopy\qty\SI} 

% --- 4. GRÁFICOS Y REFERENCIAS ---
\usepackage{graphicx}
\usepackage{xcolor}
\usepackage{tikz}
\usepackage{tcolorbox}

% Bibliografía
\usepackage[style=phys, backend=biber]{biblatex}
\addbibresource{ref.bib} 

% Hipervínculos (Siempre al final)
\usepackage[colorlinks=true, linkcolor=black, urlcolor=blue]{hyperref}
\usepackage[spanish]{cleveref} 

% --- 5. TUS COMANDOS ---
\pagestyle{fancy}
\fancyhf{}
\fancyhead[L]{Marcos Bermúdez Cobarro}
\fancyhead[R]{\thepage}

\numberwithin{equation}{section}
\newcommand{\ddt}[1]{\frac{d#1}{dt}}
\newcommand{\pdd}[2]{\frac{\partial #1}{\partial #2}}